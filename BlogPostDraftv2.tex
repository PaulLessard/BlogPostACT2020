%% LyX 2.3.5.2 created this file.  For more info, see http://www.lyx.org/.
%% Do not edit unless you really know what you are doing.
\documentclass[oneside,english]{amsart}
\usepackage[T1]{fontenc}
\usepackage[latin9]{inputenc}
\usepackage{mathrsfs}
\usepackage{amstext}
\usepackage{amsthm}
\usepackage[all]{xy}

\makeatletter
%%%%%%%%%%%%%%%%%%%%%%%%%%%%%% Textclass specific LaTeX commands.
\numberwithin{equation}{section}
\numberwithin{figure}{section}
\theoremstyle{plain}
\newtheorem{thm}{\protect\theoremname}
\theoremstyle{definition}
\newtheorem{defn}[thm]{\protect\definitionname}
\theoremstyle{remark}
\newtheorem{rem}[thm]{\protect\remarkname}

%%%%%%%%%%%%%%%%%%%%%%%%%%%%%% User specified LaTeX commands.
% Trying to deal with the \undertilde problem
% \usepackage{\undertilde}
% \input\{\undertilde}

%trying to do tikz-cd
\usepackage{tikz}
\usepackage{tikz-cd}
\usepackage{xargs}
\usetikzlibrary{arrows.meta}

%Trying to add "Yo" from Hiragana


\newcommand{\Yo}{\text{\usefont{U}{min}{m}{n}\symbol{'110}}}

\DeclareFontFamily{U}{min}{}
\DeclareFontShape{U}{min}{m}{n}{<-> dmjhira}{}


%trying to do triple arrows
\newlength{\myline}           % line thickness
\setlength{\myline}{1pt}      %setting it I geuss
\newcommandx*{\triplearrow}[4][1=0, 2=1]{
	% #1 = shorten left (optional), #2 = shorten right (optionsl),
	% #3 = draw options (must contain arrow type), #4 = path
  	\draw[line width=\myline,double distance=3\myline,#3] #4;
  	\draw[line width=\myline,shorten <=#1\myline,shorten >=#2\myline,#3] #4;
}


% define new commands for diagrams
\newcommand{\xyR}[1]{
  \xydef@\xymatrixrowsep@{#1}}
\newcommand{\xyC}[1]{
  \xydef@\xymatrixcolsep@{#1}}
\newcommand{\pullbackcorner}[1][dr]{
  \save*!/#1-1.2pc/#1:(1,-1)@^{|-}\restore}
\newcommand{\pushoutcorner}[1][dr]{
  \save*!/#1+1.2pc/#1:(-1,1)@^{|-}\restore}



\usepackage{babel}


\makeatother

\usepackage{babel}
\providecommand{\definitionname}{Definition}
\providecommand{\remarkname}{Remark}
\providecommand{\theoremname}{Theorem}

\begin{document}
%ancient paul macros...leave them...
%%%%%%%%%%%%%%%%%%%%%%%%%%%%%%%% Roman Characters %%%%%%%%%%%%%%%%%%%%%%%%%%%%%
\begin{quotation}
\global\long\def\bA{\mathbf{A}}%
\global\long\def\sA{\mathscr{A}}%
\global\long\def\sB{\mathscr{B}}%
\global\long\def\C{\mathbf{C}}%
\global\long\def\bC{\mathbb{C}}%
\global\long\def\sC{\mathscr{C}}%
\global\long\def\sfC{\mathsf{C}}%
\global\long\def\sD{\mathscr{D}}%
\global\long\def\sfD{\mathsf{D}}%
\global\long\def\sE{\mathscr{E}}%
\global\long\def\bF{\mathbb{F}}%
\global\long\def\cF{\mathcal{F}}%
\global\long\def\sF{\mathscr{F}}%
\global\long\def\bG{\mathbf{G}}%
\global\long\def\bbG{\mathbb{G}}%
%\global\long\def\cG{\mathcal{G}}%

\global\long\def\sG{\mathscr{G}}%
\global\long\def\sH{\mathscr{H}}%
\global\long\def\cI{\mathcal{I}}%
\global\long\def\sI{\mathcal{\mathscr{I}}}%
\global\long\def\fI{\mathfrak{I}}%
\global\long\def\bJ{\mathbf{J}}%
\global\long\def\cL{\mathcal{L}}%
\global\long\def\sL{\mathscr{L}}%
\global\long\def\fm{\mathfrak{m}}%
\global\long\def\sM{\mathscr{M}}%
\global\long\def\N{\mathbf{N}}%
\global\long\def\fN{\mathfrak{N}}%
\global\long\def\sN{\mathscr{N}}%
\global\long\def\cO{\mathcal{O}}%
\global\long\def\sO{\mathscr{O}}%
\global\long\def\bP{\mathbf{P}}%
\global\long\def\fp{\mathfrak{p}}%
\global\long\def\bQ{\mathbb{Q}}%
\global\long\def\fq{\mathfrak{q}}%
\global\long\def\sR{\mathscr{R}}%
\global\long\def\R{\mathbb{\mathbf{R}}}%
\global\long\def\RR{\overline{\sR}}%
\global\long\def\cS{\mathscr{\mathcal{S}}}%
\global\long\def\fS{\mathfrak{S}}%
\global\long\def\sS{\mathcal{\mathscr{S}}}%
\global\long\def\bU{\mathbf{U}}%
\global\long\def\sU{\mathfrak{\mathscr{U}}}%
\global\long\def\bV{\mathbf{V}}%
\global\long\def\sV{\mathscr{V}}%
\global\long\def\sfV{\mathsf{V}}%
\global\long\def\sW{\mathscr{W}}%
\global\long\def\sX{\mathcal{\mathscr{X}}}%
\global\long\def\bZ{\mathbb{\mathbf{Z}}}%
\global\long\def\Z{\mathbf{Z}}%
\global\long\def\frZ{\mathfrak{Z}}%
%%%%%%%%%%%%%%%%%%%%%%%%%%%%%%%% Greek Character Macros %%%%%%%%%%%%%%%%%%%%%%%%%%%%%%%%%%%%%%%

\global\long\def\e{\text{\ensuremath{\varepsilon}}}%
\global\long\def\G{\Gamma}%
%%%%%%%%%%%%%%%%%%%%%%%%%%%%%%%% Miscellaneous Character Macros %%%%%%%%%%%%%%%%%%%%%%%%%%%%%%%

\global\long\def\p{\mathcal{\prime}}%
\global\long\def\srn{\sqrt{-5}}%
%%%%%%%%%%%%%%%%%%%%%%%%%%%%%%%% Decoration Abbreviation %%%%%%%%%%%%%%%%%%%%%%%%%%%%%%%

\global\long\def\wh#1{\widehat{#1}}%
%%%%%%%%%%%%%%%%%%%%%%%%%%%%%%%% Miscellaneous Relational Macros %%%%%%%%%%%%%%%%%%%%%%%%%%%%%%

\global\long\def\nf#1#2{\nicefrac{#1}{#2}}%
%%%%%%%%%%%%%%%%%%%%%%%%%%%%%%%% Set Theoretic Macros %%%%%%%%%%%%%%%%%%%%%%%%%%%%%%%%%%%%%%%%%

\global\long\def\E{\mathsf{Ens}}%
\global\long\def\Fin{\mathsf{Fin}}%
\global\long\def\Ord{\mathsf{Ord}}%
\global\long\def\S{\mathsf{Set}}%
\global\long\def\sd#1{\left.#1\right|}%
%%%%%%%%%%%%%%%%%%%%%%%%%%%%%%%% Category Theoretic Macros %%%%%%%%%%%%%%%%%%%%%%%%%%%%%%%%%%%%

\global\long\def\Aut{\text{\ensuremath{\mathsf{Aut}}}}%
\global\long\def\Arr#1{\mathsf{Arr}\left(#1\right)}%
\global\long\def\Cat{\mathsf{Cat}}%
\global\long\def\coker{\mathrm{coker}}%
\global\long\def\colim{\underset{\longrightarrow}{\lim}}%
\global\long\def\lcolim#1{\underset{#1}{\colim}}%
\global\long\def\End{\mathsf{End}}%
\global\long\def\Ext{{\rm Ext}}%
\global\long\def\id{\text{id}}%
\global\long\def\im{\text{im}}%
\global\long\def\Im{\text{Im}}%
\global\long\def\iso{\overset{\sim}{\rightarrow}}%
\global\long\def\osi{\overset{\sim}{\leftarrow}}%
\global\long\def\liso{\overset{\sim}{\longrightarrow}}%
\global\long\def\losi{\overset{\sim}{\longleftarrow}}%
\global\long\def\Hom{\mathsf{Hom}}%
\global\long\def\BHom{\mathbf{Hom}}%
\global\long\def\Homc{\text{\text{Hom}}_{\mathscr{C}}}%
\global\long\def\Homd{\text{\text{Hom}}_{\mathscr{D}}}%
\global\long\def\Homs{\text{\text{Hom}}_{\mathsf{Set}}}%
\global\long\def\Homt{\text{\text{Hom}}_{\mathsf{Top}}}%
\global\long\def\clim{\underset{\longleftarrow}{\lim}}%
\global\long\def\cllim#1{\underset{#1}{\clim}}%
\global\long\def\Mono{\mathsf{Mono}}%
\global\long\def\Mor{\text{\ensuremath{\mathsf{Mor}}}}%
\global\long\def\Ob{\mathsf{Ob}}%
\global\long\def\one{\mathbf{1}}%
\global\long\def\op{\mathsf{op}}%
\global\long\def\Pull{\mathsf{Pull}}%
\global\long\def\Push{\mathsf{Push}}%
\global\long\def\Psh#1{\mathsf{Psh}\left(#1\right)}%
\global\long\def\pt{\text{pt.}}%
\global\long\def\Sh#1{\mathsf{Sh}\left(#1\right)}%
\global\long\def\sh{\text{sh}}%
\global\long\def\Tor{\text{\text{Tor}}}%
\global\long\def\XoY{\nicefrac{X}{Y}}%
\global\long\def\XtY{X\times Y}%
\global\long\def\Yon{\Yo}%
%%%%%%%%%%%%%%%%%%%%%%%%%%%%%%%% Algebra Macros %%%%%%%%%%%%%%%%%%%%%%%%%%%%%%%%%%%%%%%%%%%%%%

\global\long\def\Ab{\mathsf{Ab}}%
\global\long\def\AbGrp{\text{\ensuremath{\mathsf{AbGrp}}}}%
\global\long\def\alg{\mathsf{alg}}%
\global\long\def\ann{\text{ann}}%
\global\long\def\Ass{{\rm Ass}}%
\global\long\def\Ch{\mathsf{Ch}}%
\global\long\def\CR{\mathsf{ComRing}}%
\global\long\def\Gal{\text{Gal}}%
\global\long\def\Grp{\mathsf{Grp}}%
\global\long\def\Inn{\mathsf{Inn}}%
\global\long\def\Frac{\text{Frac}}%
\global\long\def\mod{\text{\ensuremath{\mathsf{mod}}}}%
\global\long\def\norm{\text{Norm}}%
\global\long\def\Norm{\mathsf{Norm}}%
\global\long\def\Orb{\mathsf{Orb}}%
\global\long\def\rad{\text{\text{rad}}}%
\global\long\def\Ring{\mathbb{\mathsf{Ring}}}%
\global\long\def\sgn{\text{sgn}}%
\global\long\def\Stab{\mathsf{Stab}}%
\global\long\def\Syl{\text{Syl}}%
\global\long\def\ZpZ{\nicefrac{\mathbb{\mathbf{Z}}}{p\bZ}}%
%%%%%%%%%%%%%%%%%%%%%%%%%%%%%%%% Homological Algebra Macros %%%%%%%%%%%%%%%%%%%%%%%%%%%%%%%%%%%

\global\long\def\CompR{_{R}\mathsf{Comp}}%
%%%%%%%%%%%%%%%%%%%%%%%%%%%%%%%% General Geometry %%%%%%%%%%%%%%%%%%%%%%%%%%%%%%%%%%%%%%%%%%%%%

\global\long\def\clm{\RR=\left(\sR,\mathbf{U},\cL,\bA\right)}%
\global\long\def\ebar{\overline{\sE}}%
\global\long\def\lrs{\mathsf{LocRngSpc}}%
\global\long\def\LRS{\mathsf{LRS}}%
\global\long\def\sHom{\mathscr{H}om}%
\global\long\def\RRing{\overline{\sR}\mathsf{-Ring}}%
%%%%%%%%%%%%%%%%%%%%%%%%%%%%%%%% Algebro-Geometric Macros %%%%%%%%%%%%%%%%%%%%%%%%%%%%%%%%%%%%

\global\long\def\Ae{\nicefrac{A\left[\varepsilon\right]}{\left(\e^{2}\right)}}%
\global\long\def\Aff{\mathsf{Aff}}%
\global\long\def\AX{\nicefrac{\Aff}{X}}%
\global\long\def\AY{\nicefrac{\Aff}{Y}}%
\global\long\def\Der{\text{Der}}%
\global\long\def\et{\text{\ensuremath{\acute{e}}t}}%
\global\long\def\etale{\acute{\text{e}}\text{tale}}%
\global\long\def\Et{\mathsf{\acute{E}t}}%
\global\long\def\ke{\nf{k\left[\varepsilon\right]}{\varepsilon^{2}}}%
\global\long\def\Pic{\text{Pic}}%
\global\long\def\proj{\text{Proj}}%
\global\long\def\Qcoh#1{\mathsf{QCoh}\left(#1\right)}%
\global\long\def\rad{\text{\text{rad}}}%
\global\long\def\red{\text{red.}}%
\global\long\def\aScheme#1{\left(\spec\left(#1\right),\cO_{\spec\left(#1\right)}\right)}%
\global\long\def\resScheme#1#2{\left(#2,\sd{\cO_{#1}}_{#2}\right)}%
\global\long\def\Sch{\mathbb{\mathsf{Sch}}}%
\global\long\def\scheme#1{\left(#1,\cO_{#1}\right)}%
\global\long\def\SCR{\S^{\CR}}%
\global\long\def\Schs{\mathsf{\nicefrac{\Sch}{S}}}%
\global\long\def\ShAb#1{\mathsf{Sh}_{\AbGrp}\left(#1\right)}%
\global\long\def\Spec{\text{Spec}}%
\global\long\def\Sym{\text{Sym}}%
\global\long\def\Zar{\mathsf{Zar}}%
%%%%%%%%%%%%%%%%%%%%%%%%%%%%%%%% Analysis Macros %%%%%%%%%%%%%%%%%%%%%%%%%%%%%%%%%%%%%%%%%%%%%%%

\global\long\def\co#1#2{\left[#1,#2\right)}%
\global\long\def\oc#1#2{\left(#1,#2\right]}%
\global\long\def\loc{\mathsf{\text{loc.}}}%
\global\long\def\nn#1{\left\Vert #1\right\Vert }%
\global\long\def\Re{\text{Re}}%
\global\long\def\supp{\text{supp}}%
\global\long\def\XSM{\left(X,\cS,\mu\right)}%
%%%%%%%%%%%%%%%%%%%%%%%%%%%%%%%% Differential Macros %%%%%%%%%%%%%%%%%%%%%%%%%%%%%%%%%%%%%%%%%%%

\global\long\def\grad{\text{grad}}%
\global\long\def\Homrv{\text{\text{Hom}}_{\mathbb{R}-\mathsf{Vect}}}%
%%%%%%%%%%%%%%%%%%%%%%%%%%%%%%%% Differential Geometric Macros %%%%%%%%%%%%%%%%%%%%%%%%%%%%%%%%%

%%%%%%%%%%%%%%%%%%%%%%%%%%%%%%%% Topology Macros %%%%%%%%%%%%%%%%%%%%%%%%%%%%%%%%%%%%%%%%%%%%%%%

\global\long\def\Cech{{\rm \check{C}ech}}%
\global\long\def\cCC{\check{\mathcal{C}}}%
\global\long\def\CC{\check{C}}%
\global\long\def\CH{\mathsf{CHaus}}%
\global\long\def\Cov{\mathsf{Cov}}%
\global\long\def\CW{\mathsf{CW}}%
\global\long\def\HT{\mathsf{HTop}}%
\global\long\def\Homt{\text{\text{Hom}}_{\mathsf{Top}}}%
\global\long\def\Homrv{\text{\text{Hom}}_{\mathbb{R}-\mathsf{Vect}}}%
\global\long\def\MT{\text{\ensuremath{\mathsf{Mor}}}_{\T}}%
\global\long\def\Open{\text{\ensuremath{\mathsf{Open}}}}%
\global\long\def\PT{\mathsf{P-Top}}%
\global\long\def\T{\mathsf{Top}}%
% Homotopy Theory specific macros

\global\long\def\Ad{\mathsf{Ad}}%
\global\long\def\Cell{\mathsf{Cell}}%
\global\long\def\Cov{\mathsf{Cov}}%
\global\long\def\Sp{\mathsf{Sp}}%
\global\long\def\Spectra{\mathsf{Spectra}}%
\global\long\def\ss{\widehat{\triangle}}%
\global\long\def\Tn{\mathbb{T}^{n}}%
\global\long\def\Sk#1{\textrm{Sk}^{#1}}%
\global\long\def\smash{\wedge}%
\global\long\def\wp{\vee}%
% HoTT specific macros

\global\long\def\base{\mathsf{base}}%
\global\long\def\comp{\mathsf{comp}}%
\global\long\def\funext{\mathsf{funext}}%
\global\long\def\hfib{\text{\ensuremath{\mathsf{hfib}}}}%
\global\long\def\I{\mathbf{I}}%
\global\long\def\ind{\mathsf{ind}}%
% LOOP as \mathsf WHAT THE FUCK!!!!

\global\long\def\lp{\mathsf{loop}}%
\global\long\def\pair{\mathsf{pair}}%
\global\long\def\pr{\mathbf{\mathsf{pr}}}%
\global\long\def\rec{\mathsf{rec}}%
\global\long\def\refl{\mathsf{refl}}%
\global\long\def\transport{\mathsf{transport}}%
%%%%%%%%%%%%%%%%%%%%%%%%%%%%%%%% UNSORTED! %%%%%%%%%%%%%%%%%%%%%%%%%%%%%%%%%%%%%%%%%%%%%%%%%%%%%

\global\long\def\is{\triangle\raisebox{2mm}{\mbox{\ensuremath{\infty}}}}%
\global\long\def\Cof{\mathsf{Cof}}%
\global\long\def\sfW{\mathsf{W}}%
\global\long\def\Cyl{\mathsf{Cyl}}%
\global\long\def\Mono{\mathsf{Mono}}%
\global\long\def\t{\triangle}%
\global\long\def\tl{\triangleleft}%
\global\long\def\tr{\triangleright}%
\global\long\def\Shift{\mathrm{Shift}_{+1}}%
\global\long\def\Shiftd{\mathrm{Shift}_{-1}}%
\global\long\def\out{\mathrm{out}}%
\global\long\def\cN{\mathcal{N}}%
\global\long\def\fC{\mathcal{\mathfrak{C}}}%
\global\long\def\ev{\mathsf{ev}}%
\global\long\def\Map{\mathsf{Map}}%
\global\long\def\whp#1{\wh{#1}_{\bullet}}%
\global\long\def\bfTwo{\mathbf{2}}%
\global\long\def\bfL{\mathbf{L}}%
\global\long\def\bfR{\mathbf{R}}%
\global\long\def\sJ{\mathscr{J}}%
\global\long\def\Sing{\mathsf{Sing}}%
\global\long\def\Sph{\mathsf{Sph}}%
\global\long\def\whfin#1{\widehat{#1}_{\mathrm{fin}}}%
\global\long\def\whpfin#1{\widehat{#1}_{\mathrm{\bullet fin}}}%
\global\long\def\fin{\mathsf{fin}}%
\global\long\def\cT{\mathcal{T}}%
\global\long\def\Alg{\mathsf{Alg}}%
\global\long\def\st{\mathsf{st}}%
\global\long\def\IN{\mathsf{in}}%
\global\long\def\hr{\mathsf{hr}}%
\global\long\def\Fun{\mathsf{Fun}}%
\global\long\def\Th{\mathsf{Th}}%
\global\long\def\sT{\mathscr{T}}%
\global\long\def\Lex{\mathsf{Lex}}%
\global\long\def\FinSet{\mathsf{FinSet}}%
\global\long\def\Fib{\mathsf{Fib}}%
\global\long\def\FPS{\mathsf{FinPos}}%
\global\long\def\Mod{\mathsf{Mod}}%
\global\long\def\sfA{\mathsf{A}}%
\global\long\def\sfV{\mathsf{V}}%
\global\long\def\inner{\mathsf{inner}}%
\global\long\def\bfI{\mathbf{I}}%
\global\long\def\Kan{\mathsf{Kan}}%
\global\long\def\Berger{\mathsf{Berger}}%
\end{quotation}
%ancient paul macros...leave them...

%macros for free prop section
\global\long\def\fF{\mathfrak{F}}%
 
\global\long\def\cG{\mathcal{G}}%
 
\global\long\def\cR{\mathcal{R}}%
 
\global\long\def\cD{\mathcal{D}}%
 %\newcommand{\Ob}{\mathsf{Ob}}
\global\long\def\evmap{\mathsf{ev}}%
 
\global\long\def\coev{\mathsf{coev}}%
 
\global\long\def\bigslant#1#2{{\raisebox{.2em}{$#1$}\left/\raisebox{-.2em}{$#2$}\right.}}%


\subsection{From domain to foundations}

While in the cases of $\mathsf{HoTT}$ or intuitionistic linear logic
we have a particular form of reason or computation we wish to model,
and only afterwards are we trying to use these type theories to reason
\emph{about }$\left(\infty,1\right)$-toposes or closed symmetric
monoidal categories, Shulman's project with practical type theory
is to start with a domain and reverse engineer a syntax; we have a
sort of category, symmetric monoidal categories, and wish to ask them
to provide us with a type theory. This is not to say however that
we have \emph{no} hand in specifying the type theory.

\subsection{Criteria for practicality}
\begin{itemize}
\item syntax semantics correspondence satisfying the following criteria... 
\end{itemize}
First, for any symmetric monoidal category, not merely concrete and
familiar ones, the type theory we get should permit us to leverage
our intuition and experience with reasoning about sets with elements\footnote{A possibly enlightening parallel to this program can be found in the
history of algebraic geometry. Just as the Zariski topology extends
the interpretation, beyond the rings of polynomial functions on algebraic
varieties, of rings as rings of functions on spaces, a philosophical
perspective on the goal of Shulman's practical type theory extends
the formal metaphor of symmetric monoidal categories of vector spaces
to arbitary symetric monoidal categories.}. As such our type theory and its semantics should correlate 
\begin{eqnarray*}
\mathrm{objects} & \longleftrightarrow & \mathrm{contexts}\\
\mathrm{morphisms} & \longleftrightarrow & \mathrm{typing\ judgements}
\end{eqnarray*}

Second, we must not: 
\begin{quotation}
``(strike) a Map of the Empire whose size was that of the Empire''
- J.L.Borges ``On Exactitude in Science''\footnote{https://kwarc.info/teaching/TDM/Borges.pdf} 
\end{quotation}
so we ask that this type theory be specified not by a symmetric monoidal
categories directly but instead by elegant presentations thereof.

Lastly, we wish to be assured that the theory is complete in the sense
that everything which holds in a particular symmetric monoidal category
is derivable as a judgement.

\subsection{Sweedler's notation for co-algebras and Shulman's term judgements}

We cannot simply bend some cartesian type theory to our will to develop
a practical type theory for symmetric monoidal categories, because,
while $\times$ enjoys a universal property characterized in terms
of projection maps, the universal property of $\otimes$ is characterized
in terms of co-projection maps. So, while in a cartesian monoidal
category like $\S$, for any $z\in X\times Y$, we have that\footnote{Type theoretically, this is precisely the derived judgement of the
$\eta$-computation rule for the product type.} 
\[
z=\mathsf{pair}\left(\pr_{1}\left(z\right),\pr_{2}\left(z\right)\right)
\]
no such characterization of the elements of $X\otimes Y$ is immediately
available.

Locating the concern in the familiar terrain of vector spaces provides
this categorical phenomenon with concrete intuition: the tensor product
$\bigotimes_{i\in\left\{ 1,\dots,n\right\} }V_{i}$ of vector spaces
is comprised not of simple tensors 
\[
v=v_{1}\otimes\cdots\otimes v_{n}=\mathsf{tuple}\left(v_{1},\dots,v_{n}\right)
\]
but of linear combinations of them. We can however pretend that every
tensor is a tuple, provided we are very careful; this is exactly the
power of Sweedler's notation for co-algebras which Shulman's expands
into a syntax for symmetric monoidal categories.

The co-multiplication of a co-algebra in vector spaces may be written
elementarily as follows. 
\[
\vcenter{\vbox{\xyR{0pc}\xyC{3pc}\xymatrix{C\ar[r]^{\t} & C\otimes C\\
c\ar@{|->}[r] & \sum_{i=1}^{k}c_{\left(1\right)}^{i}\otimes c_{\left(2\right)}^{i}
}
}}
\]
Sweedler long ago noted that it was convenient to drop summation and
even to drop indices. Going further and replacing $\left(\_\right)\otimes\cdots\otimes\left(\_\right)$
with $\left(\_,\dots,\_\right)$, both in the formation of tensor
products and for 'simple tensors', we then denote maps into arbitrary
tensor products as follows. 
\[
\vcenter{\vbox{\xyR{0pc}\xyC{3pc}\xymatrix{A\ar[r]^{f} & \left(B_{1},\dots,B_{n}\right)\\
a\ar@{|->}[r] & \left(f_{\left(1\right)}\left(a\right),\dots,f_{\left(n\right)}\left(a\right)\right)
}
}}
\]
Shulman translates this notation into term formation rules: for example\footnote{For expository purposes we've replaced $f\in\mathcal{G}\left(A_{1},\dots,A_{m};B_{1},\dots,B_{m}\right)$,
which is the hypothesis that $f$ is a \emph{generating }morphism,
with the more general hypothesis that $f\in\Hom\left(A_{1},\dots,A_{m};B_{1},\dots,B_{n}\right)$.} 
\[
\frac{\begin{array}{c}
\Gamma\vdash M_{1}\ \mathsf{term}\dots\Gamma\vdash M_{m}\ \mathsf{term}\\
f\in\Hom\left(A_{1},\dots,A_{m};B_{1},\dots,B_{n}\right)\ m\geq1\ n\geq2\ 1\leq k\leq n
\end{array}}{\Gamma\vdash f_{\left(k\right)}\left(M_{1},\dots,M_{m}\right)}
\]
which in the case $m=1$ and $\Gamma\vdash a\ \mathsf{term}$ allows
us to form the bare terms $f_{\left(k\right)}\left(a\right)$ as in
Sweedler's notation and then the rules for our typing judgements,
which we'll cover later, will allow us to derive 
\[
a:A\vdash\left(f_{\left(1\right)}\left(a\right),\dots,f_{\left(k\right)}\left(a\right)\right):\left(B_{1},\dots,B_{n}\right)
\]
the typing judgement which corresponds to the morphism $f$.

What's more, just as the syntactic limits on defineable expressions
in $\mathsf{HoTT}$ grant that any expression is born invariant, the
term formation rules of our type theory will make sure that only permissable
expressions using Sweedler's notation will be syntactically accessible.

\subsection{A suitable choice of presentation}

It is 'well known' that every symmetric monoidal category is equivalent
to a symmetric strict monoidal category. Signifigantly less well known
is that every symmetric strict monoidal category is equivalent to
a PROP.
\begin{defn}
A \textbf{PROP} $\left(P,\mathscr{P}\right)$ is comprised of:
\begin{itemize}
\item a set $P$ of generating objects; and
\item a symmetric strict monoidal category $\mathscr{P}=\left(\mathscr{P},\otimes,\mathbf{1}\right)$
for which $\left(\Ob\left(\mathscr{P}\right),\otimes,\mathbf{1}\right)$
is the free commutative monoid on the set $P$.
\end{itemize}
A morphism of props is a pair $\left(f,F\right):\left(P,\mathscr{P}\right)\longrightarrow\left(Q,\mathscr{Q}\right)$
where $f:P\longrightarrow Q$ is a function and $F:\mathscr{P}\longrightarrow\mathscr{Q}$
is a strong symmetric monoidal functor with $\Ob\left(F\right)$ being
the morphism of commutative monoids induced by the function $f$.
We denote the category of PROPs by $\mathsf{PROP}$.
\end{defn}

Indeed we've an equivalence of categories
\[
\mathsf{SymStrMonCat_{strong}\liso\mathsf{PROP}}
\]
The utility of this observation for our ends is two-fold:
\begin{itemize}
\item syntactically, finite lists up to symmetry are far si
\item a presentation of a $\mathsf{PROP}$ need only be comprised of three
sorts of data:
\begin{itemize}
\item a generating set of objects;
\item a generating set of morphisms, formal arrows between finite lists
of generating objects;
\item a generating sets of equalities of morphisms, between formal composites
and tensors generating morphism.
\end{itemize}
\end{itemize}
An elegant packaging of this second fact follows.
\begin{defn}
Let a \textbf{signature }for a prop $\mathcal{G}=\left(G_{0},G_{1}\right)$
be comprised of a pair of sets: 
\begin{itemize}
\item $G_{0}$ a generating set of objects; and
\item $G_{1}$ a set of formal arrows $f:\left(A_{1},\dots,A_{n}\right)\longrightarrow\left(B_{1},\dots,B_{m}\right)$
where the $A_{i}$ and $B_{j}$ are of $G_{0}$. 
\end{itemize}
\end{defn}

\begin{rem}
AThose familiar with the theory of computads will note that the data
enumerated above are exactly a 2-computad for a PROPs.
\end{rem}


\subsection{On initiality and the admissability of structural rules}

In this framework the completeness result desired takes the form of
an \emph{initiality theorem}, a theorem which here will posit that:
\begin{itemize}
\item for $\sC$ be a symmetric monoidal category and $\mathbf{T}_{\sC}$
the associated type theory,
\item the term model of $\mathbf{T}_{\sC}$, the initial semantic interpretation
of $\mathbf{T}_{\sC}$, is equivalent as a symmetric monoidal category
to $\sC$. 
\end{itemize}
In particular this 

\section{Architecture of the paper}

\subsection{Idea of the content of the paper}

What is the general idea for obtaining this practical type theory?\\
 
\begin{enumerate}
\item \textit{Start with some input data (which we will call signature)} 
\begin{itemize}
\item Our input data for constructing type theories will consist of a set
of objects together with a set of arrows whose domain and codomain
consist of finite lists of objets. 
\end{itemize}
\item \textit{We build a type theory for the free prop generated by a signature} 
\begin{itemize}
\item This is done by defining rules for terms and rules for typing judgements. 
\item Much care is taken while defining these rules to ensure, among other
things, that the composition and the exchange rules are admissible,
therefore ensuring that any judgement has a unique derivation. This
is a key requirement to prove the initiality theorem that we discussed
earlier. 
\end{itemize}
\item \textit{The term model of this type theory can be proven to be the
prop freely generated by the input signature. Hence we now have the
initiality theorem} %\item modulo by equality rule

\begin{itemize}
\item Taking the contexts of this type theory as objects and the derivable
term judgement (modulo an equality rule) as morphisms forms a strict
symmetric monoidal category, which we denote $\fF\cG$. It is also
easy to show that $\fF\cG$ is in fact also a prop. 
\item We can then show that $\fF\cG$ is actually the free prop generate
by $\cG$. 
\end{itemize}
Great. Now we have a type theory for props freely generated by a signature.
But, what about all the other props? Or in other words, how do we
deal with SMC that have additional equality relations? Well, since
we know that the category of prop is monadic over the category of
signatures, we have that every prop $\mathcal{P}$ admit a presentation
in terms of signatures (i.e. a coequaliser diagram $\fF\cR\rightrightarrows\fF\cG\rightarrow\mathcal{P}$),
where $\cR$ is a signature that provides equality axioms. 
\item \textit{We then essentially redo the two previous steps, but this
time we also quotient by these additional equality axioms provided
by the signature $\cR$ in the presentation of the prop} 
\end{enumerate}
That's it! Following these steps gives us a type theory for the prop
presented by $(\cG,\cR)$, which, as proven in the paper, allows us
to reason about structures in any props, and hence also in any symmetric
monoidal category.\\


\subsection{Illustrating the content of the paper using the free dual pair example}

Start with the duality $(D,D^{*},\textsf{ev},\textsf{coev})$ (and
its associated free dual pair $\cD$) given above.
\begin{enumerate}
\item \textit{Start with some input data} 
\begin{itemize}
\item Starting with the duality $(D,D^{*},\textsf{ev},\textsf{coev})$,
we define some input data (signature) $\cG$ as follows: 
\begin{itemize}
\item Objects: $\{D,D^{*}\}$ 
\item Arrows: $\{\msf{ev}:(D^{*},D)\to(),\quad\msf{coev}:()\to(D,D^{*})\}$ 
\end{itemize}
\end{itemize}
\item \textit{Build a type theory from the input data} 
\begin{itemize}
\item The arrows in the input data become the rules in the type theory $\mathbf{T}_{\cG}$:

\begin{minipage}[c]{0.42\textwidth}%
 
\[
\begin{tikzcd}[ampersandreplacement=\&,rowsep=small]\mathsf{ev}:(D^{\star},D)\longrightarrow()\arrow[d,mapsto]\end{tikzcd}
\]
%
\end{minipage}%
\begin{minipage}[c]{0.42\textwidth}%
 
\[
\begin{tikzcd}[ampersandreplacement=\&,rowsep=small]\mathsf{coev}:()\longrightarrow(D^{\star},D)\arrow[d,mapsto]\end{tikzcd}
\]
%
\end{minipage}
\end{itemize}
{*}{*}{*}HOW CAN WE EXPRESS THE EQUALITY RULE IN THIS EXAMPLE? 
\item \textit{Prove initiality theorem} 
\begin{itemize}
\item By the theorems in the paper the context and derivable typing judgements
in the type theory $\mathbf{T}_{\cG}$ form a prop which is in fact
the free prop generated by the input data $\cG$. 
\end{itemize}
\item \textit{Account for equality relations by using a presentation as
a coequaliser} 
\begin{itemize}
\item The free dual pair $\cD$ admits a presentation as the following colimit:
\begin{align*}
\underset{\longrightarrow}{\textrm{lim}}\left\{ \fF\cR\rightrightarrows\fF\cG\right\} \overset{\sim}{\longrightarrow}\cD
\end{align*}
where $R$ is the signature of relations that imposes the following
two axioms (these correspond to the commutative diagrams introduced
above): 
\begin{align*}
(\mathsf{ev}\otimes\mathsf{id}_{D})\circ(\mathsf{id}_{D}\otimes\mathsf{coev})=\mathsf{id}_{D}\qquad\qquad(\mathsf{ev}\otimes\mathsf{id}_{D})\circ(\mathsf{id}_{D}\otimes\mathsf{coev})=\mathsf{id}_{D}
\end{align*}

First translation: 
\begin{align*}
x:D\vdash(\eta_{(1)}\;|\;\varepsilon(\eta_{(2)},x))=x:D\qquad y:D^{*}\vdash(\eta_{(2)}\;|\;\varepsilon(y,\eta_{(1)}))=y:D^{*}
\end{align*}

Second translation: 
\begin{align*}
x:D\vdash(u\;|\;\lambda^{D}u\triangleleft x)=x:D\qquad y:D^{*}\vdash(\lambda^{D}u\;|\;y\triangleleft x)=D^{*}
\end{align*}
{*}{*}{*}NEEDS FIXING!!! 
\end{itemize}
\end{enumerate}

\end{document}
