%% LyX 2.3.5.2 created this file.  For more info, see http://www.lyx.org/.
%% Do not edit unless you really know what you are doing.
\documentclass[oneside,english]{amsart}
\usepackage[T1]{fontenc}
\usepackage[latin9]{inputenc}
\usepackage{mathrsfs}
\usepackage{amstext}
\usepackage{amsthm}
\usepackage[all]{xy}

\makeatletter
%%%%%%%%%%%%%%%%%%%%%%%%%%%%%% Textclass specific LaTeX commands.
\numberwithin{equation}{section}
\numberwithin{figure}{section}
\theoremstyle{plain}
\newtheorem{thm}{\protect\theoremname}
\theoremstyle{definition}
\newtheorem{defn}[thm]{\protect\definitionname}
\theoremstyle{plain}
\newtheorem{lem}[thm]{\protect\lemmaname}

%%%%%%%%%%%%%%%%%%%%%%%%%%%%%% User specified LaTeX commands.
% Trying to deal with the \undertilde problem
% \usepackage{\undertilde}
% \input\{\undertilde}

%trying to do tikz-cd
\usepackage{tikz}
\usepackage{tikz-cd}
\usepackage{xargs}
\usetikzlibrary{arrows.meta}

%adding prooftree stuff
\usepackage{prftree}

%Trying to add "Yo" from Hiragana


\newcommand{\Yo}{\text{\usefont{U}{min}{m}{n}\symbol{'110}}}

\DeclareFontFamily{U}{min}{}
\DeclareFontShape{U}{min}{m}{n}{<-> dmjhira}{}


%trying to do triple arrows
\newlength{\myline}           % line thickness
\setlength{\myline}{1pt}      %setting it I geuss
\newcommandx*{\triplearrow}[4][1=0, 2=1]{
	% #1 = shorten left (optional), #2 = shorten right (optionsl),
	% #3 = draw options (must contain arrow type), #4 = path
  	\draw[line width=\myline,double distance=3\myline,#3] #4;
  	\draw[line width=\myline,shorten <=#1\myline,shorten >=#2\myline,#3] #4;
}


% define new commands for diagrams
\newcommand{\xyR}[1]{
  \xydef@\xymatrixrowsep@{#1}}
\newcommand{\xyC}[1]{
  \xydef@\xymatrixcolsep@{#1}}
\newcommand{\pullbackcorner}[1][dr]{
  \save*!/#1-1.2pc/#1:(1,-1)@^{|-}\restore}
\newcommand{\pushoutcorner}[1][dr]{
  \save*!/#1+1.2pc/#1:(-1,1)@^{|-}\restore}



\usepackage{babel}

\makeatother

\usepackage{babel}
\providecommand{\definitionname}{Definition}
\providecommand{\lemmaname}{Lemma}
\providecommand{\theoremname}{Theorem}

\begin{document}
%ancient paul macros...leave them...
%%%%%%%%%%%%%%%%%%%%%%%%%%%%%%%% Roman Characters %%%%%%%%%%%%%%%%%%%%%%%%%%%%%
\begin{quotation}
\global\long\def\bA{\mathbf{A}}%
\global\long\def\sA{\mathscr{A}}%
\global\long\def\sB{\mathscr{B}}%
\global\long\def\C{\mathbf{C}}%
\global\long\def\bC{\mathbb{C}}%
\global\long\def\sC{\mathscr{C}}%
\global\long\def\sfC{\mathsf{C}}%
\global\long\def\sD{\mathscr{D}}%
\global\long\def\sfD{\mathsf{D}}%
\global\long\def\sE{\mathscr{E}}%
\global\long\def\bF{\mathbb{F}}%
\global\long\def\cF{\mathcal{F}}%
\global\long\def\sF{\mathscr{F}}%
\global\long\def\bG{\mathbf{G}}%
\global\long\def\bbG{\mathbb{G}}%
%\global\long\def\cG{\mathcal{G}}%

\global\long\def\sG{\mathscr{G}}%
\global\long\def\sH{\mathscr{H}}%
\global\long\def\cI{\mathcal{I}}%
\global\long\def\sI{\mathcal{\mathscr{I}}}%
\global\long\def\fI{\mathfrak{I}}%
\global\long\def\bJ{\mathbf{J}}%
\global\long\def\cL{\mathcal{L}}%
\global\long\def\sL{\mathscr{L}}%
\global\long\def\fm{\mathfrak{m}}%
\global\long\def\sM{\mathscr{M}}%
\global\long\def\N{\mathbf{N}}%
\global\long\def\fN{\mathfrak{N}}%
\global\long\def\sN{\mathscr{N}}%
\global\long\def\cO{\mathcal{O}}%
\global\long\def\sO{\mathscr{O}}%
\global\long\def\bP{\mathbf{P}}%
\global\long\def\fp{\mathfrak{p}}%
\global\long\def\bQ{\mathbb{Q}}%
\global\long\def\fq{\mathfrak{q}}%
\global\long\def\sR{\mathscr{R}}%
\global\long\def\R{\mathbb{\mathbf{R}}}%
\global\long\def\RR{\overline{\sR}}%
\global\long\def\cS{\mathscr{\mathcal{S}}}%
\global\long\def\fS{\mathfrak{S}}%
\global\long\def\sS{\mathcal{\mathscr{S}}}%
\global\long\def\bU{\mathbf{U}}%
\global\long\def\sU{\mathfrak{\mathscr{U}}}%
\global\long\def\bV{\mathbf{V}}%
\global\long\def\sV{\mathscr{V}}%
\global\long\def\sfV{\mathsf{V}}%
\global\long\def\sW{\mathscr{W}}%
\global\long\def\sX{\mathcal{\mathscr{X}}}%
\global\long\def\bZ{\mathbb{\mathbf{Z}}}%
\global\long\def\Z{\mathbf{Z}}%
\global\long\def\frZ{\mathfrak{Z}}%
%%%%%%%%%%%%%%%%%%%%%%%%%%%%%%%% Greek Character Macros %%%%%%%%%%%%%%%%%%%%%%%%%%%%%%%%%%%%%%%

\global\long\def\e{\text{\ensuremath{\varepsilon}}}%
\global\long\def\G{\Gamma}%
%%%%%%%%%%%%%%%%%%%%%%%%%%%%%%%% Miscellaneous Character Macros %%%%%%%%%%%%%%%%%%%%%%%%%%%%%%%

\global\long\def\p{\mathcal{\prime}}%
\global\long\def\srn{\sqrt{-5}}%
%%%%%%%%%%%%%%%%%%%%%%%%%%%%%%%% Decoration Abbreviation %%%%%%%%%%%%%%%%%%%%%%%%%%%%%%%

\global\long\def\wh#1{\widehat{#1}}%
%%%%%%%%%%%%%%%%%%%%%%%%%%%%%%%% Miscellaneous Relational Macros %%%%%%%%%%%%%%%%%%%%%%%%%%%%%%

\global\long\def\nf#1#2{\nicefrac{#1}{#2}}%
%%%%%%%%%%%%%%%%%%%%%%%%%%%%%%%% Set Theoretic Macros %%%%%%%%%%%%%%%%%%%%%%%%%%%%%%%%%%%%%%%%%

\global\long\def\E{\mathsf{Ens}}%
\global\long\def\Fin{\mathsf{Fin}}%
\global\long\def\Ord{\mathsf{Ord}}%
\global\long\def\S{\mathsf{Set}}%
\global\long\def\sd#1{\left.#1\right|}%
%%%%%%%%%%%%%%%%%%%%%%%%%%%%%%%% Category Theoretic Macros %%%%%%%%%%%%%%%%%%%%%%%%%%%%%%%%%%%%

\global\long\def\Aut{\text{\ensuremath{\mathsf{Aut}}}}%
\global\long\def\Arr#1{\mathsf{Arr}\left(#1\right)}%
\global\long\def\Cat{\mathsf{Cat}}%
\global\long\def\coker{\mathrm{coker}}%
\global\long\def\colim{\underset{\longrightarrow}{\lim}}%
\global\long\def\lcolim#1{\underset{#1}{\colim}}%
\global\long\def\End{\mathsf{End}}%
\global\long\def\Ext{{\rm Ext}}%
\global\long\def\id{\text{id}}%
\global\long\def\im{\text{im}}%
\global\long\def\Im{\text{Im}}%
\global\long\def\iso{\overset{\sim}{\rightarrow}}%
\global\long\def\osi{\overset{\sim}{\leftarrow}}%
\global\long\def\liso{\overset{\sim}{\longrightarrow}}%
\global\long\def\losi{\overset{\sim}{\longleftarrow}}%
\global\long\def\Hom{\mathsf{Hom}}%
\global\long\def\BHom{\mathbf{Hom}}%
\global\long\def\Homc{\text{\text{Hom}}_{\mathscr{C}}}%
\global\long\def\Homd{\text{\text{Hom}}_{\mathscr{D}}}%
\global\long\def\Homs{\text{\text{Hom}}_{\mathsf{Set}}}%
\global\long\def\Homt{\text{\text{Hom}}_{\mathsf{Top}}}%
\global\long\def\clim{\underset{\longleftarrow}{\lim}}%
\global\long\def\cllim#1{\underset{#1}{\clim}}%
\global\long\def\Mono{\mathsf{Mono}}%
\global\long\def\Mor{\text{\ensuremath{\mathsf{Mor}}}}%
\global\long\def\Ob{\mathsf{Ob}}%
\global\long\def\one{\mathbf{1}}%
\global\long\def\op{\mathsf{op}}%
\global\long\def\Pull{\mathsf{Pull}}%
\global\long\def\Push{\mathsf{Push}}%
\global\long\def\Psh#1{\mathsf{Psh}\left(#1\right)}%
\global\long\def\pt{\text{pt.}}%
\global\long\def\Sh#1{\mathsf{Sh}\left(#1\right)}%
\global\long\def\sh{\text{sh}}%
\global\long\def\Tor{\text{\text{Tor}}}%
\global\long\def\XoY{\nicefrac{X}{Y}}%
\global\long\def\XtY{X\times Y}%
\global\long\def\Yon{\Yo}%
%%%%%%%%%%%%%%%%%%%%%%%%%%%%%%%% Algebra Macros %%%%%%%%%%%%%%%%%%%%%%%%%%%%%%%%%%%%%%%%%%%%%%

\global\long\def\Ab{\mathsf{Ab}}%
\global\long\def\AbGrp{\text{\ensuremath{\mathsf{AbGrp}}}}%
\global\long\def\alg{\mathsf{alg}}%
\global\long\def\ann{\text{ann}}%
\global\long\def\Ass{{\rm Ass}}%
\global\long\def\Ch{\mathsf{Ch}}%
\global\long\def\CR{\mathsf{ComRing}}%
\global\long\def\Gal{\text{Gal}}%
\global\long\def\Grp{\mathsf{Grp}}%
\global\long\def\Inn{\mathsf{Inn}}%
\global\long\def\Frac{\text{Frac}}%
\global\long\def\mod{\text{\ensuremath{\mathsf{mod}}}}%
\global\long\def\norm{\text{Norm}}%
\global\long\def\Norm{\mathsf{Norm}}%
\global\long\def\Orb{\mathsf{Orb}}%
\global\long\def\rad{\text{\text{rad}}}%
\global\long\def\Ring{\mathbb{\mathsf{Ring}}}%
\global\long\def\sgn{\text{sgn}}%
\global\long\def\Stab{\mathsf{Stab}}%
\global\long\def\Syl{\text{Syl}}%
\global\long\def\ZpZ{\nicefrac{\mathbb{\mathbf{Z}}}{p\bZ}}%
%%%%%%%%%%%%%%%%%%%%%%%%%%%%%%%% Homological Algebra Macros %%%%%%%%%%%%%%%%%%%%%%%%%%%%%%%%%%%

\global\long\def\CompR{_{R}\mathsf{Comp}}%
%%%%%%%%%%%%%%%%%%%%%%%%%%%%%%%% General Geometry %%%%%%%%%%%%%%%%%%%%%%%%%%%%%%%%%%%%%%%%%%%%%

\global\long\def\clm{\RR=\left(\sR,\mathbf{U},\cL,\bA\right)}%
\global\long\def\ebar{\overline{\sE}}%
\global\long\def\lrs{\mathsf{LocRngSpc}}%
\global\long\def\LRS{\mathsf{LRS}}%
\global\long\def\sHom{\mathscr{H}om}%
\global\long\def\RRing{\overline{\sR}\mathsf{-Ring}}%
%%%%%%%%%%%%%%%%%%%%%%%%%%%%%%%% Algebro-Geometric Macros %%%%%%%%%%%%%%%%%%%%%%%%%%%%%%%%%%%%

\global\long\def\Ae{\nicefrac{A\left[\varepsilon\right]}{\left(\e^{2}\right)}}%
\global\long\def\Aff{\mathsf{Aff}}%
\global\long\def\AX{\nicefrac{\Aff}{X}}%
\global\long\def\AY{\nicefrac{\Aff}{Y}}%
\global\long\def\Der{\text{Der}}%
\global\long\def\et{\text{\ensuremath{\acute{e}}t}}%
\global\long\def\etale{\acute{\text{e}}\text{tale}}%
\global\long\def\Et{\mathsf{\acute{E}t}}%
\global\long\def\ke{\nf{k\left[\varepsilon\right]}{\varepsilon^{2}}}%
\global\long\def\Pic{\text{Pic}}%
\global\long\def\proj{\text{Proj}}%
\global\long\def\Qcoh#1{\mathsf{QCoh}\left(#1\right)}%
\global\long\def\rad{\text{\text{rad}}}%
\global\long\def\red{\text{red.}}%
\global\long\def\aScheme#1{\left(\spec\left(#1\right),\cO_{\spec\left(#1\right)}\right)}%
\global\long\def\resScheme#1#2{\left(#2,\sd{\cO_{#1}}_{#2}\right)}%
\global\long\def\Sch{\mathbb{\mathsf{Sch}}}%
\global\long\def\scheme#1{\left(#1,\cO_{#1}\right)}%
\global\long\def\SCR{\S^{\CR}}%
\global\long\def\Schs{\mathsf{\nicefrac{\Sch}{S}}}%
\global\long\def\ShAb#1{\mathsf{Sh}_{\AbGrp}\left(#1\right)}%
\global\long\def\Spec{\text{Spec}}%
\global\long\def\Sym{\text{Sym}}%
\global\long\def\Zar{\mathsf{Zar}}%
%%%%%%%%%%%%%%%%%%%%%%%%%%%%%%%% Analysis Macros %%%%%%%%%%%%%%%%%%%%%%%%%%%%%%%%%%%%%%%%%%%%%%%

\global\long\def\co#1#2{\left[#1,#2\right)}%
\global\long\def\oc#1#2{\left(#1,#2\right]}%
\global\long\def\loc{\mathsf{\text{loc.}}}%
\global\long\def\nn#1{\left\Vert #1\right\Vert }%
\global\long\def\Re{\text{Re}}%
\global\long\def\supp{\text{supp}}%
\global\long\def\XSM{\left(X,\cS,\mu\right)}%
%%%%%%%%%%%%%%%%%%%%%%%%%%%%%%%% Differential Macros %%%%%%%%%%%%%%%%%%%%%%%%%%%%%%%%%%%%%%%%%%%

\global\long\def\grad{\text{grad}}%
\global\long\def\Homrv{\text{\text{Hom}}_{\mathbb{R}-\mathsf{Vect}}}%
%%%%%%%%%%%%%%%%%%%%%%%%%%%%%%%% Differential Geometric Macros %%%%%%%%%%%%%%%%%%%%%%%%%%%%%%%%%

%%%%%%%%%%%%%%%%%%%%%%%%%%%%%%%% Topology Macros %%%%%%%%%%%%%%%%%%%%%%%%%%%%%%%%%%%%%%%%%%%%%%%

\global\long\def\Cech{{\rm \check{C}ech}}%
\global\long\def\cCC{\check{\mathcal{C}}}%
\global\long\def\CC{\check{C}}%
\global\long\def\CH{\mathsf{CHaus}}%
\global\long\def\Cov{\mathsf{Cov}}%
\global\long\def\CW{\mathsf{CW}}%
\global\long\def\HT{\mathsf{HTop}}%
\global\long\def\Homt{\text{\text{Hom}}_{\mathsf{Top}}}%
\global\long\def\Homrv{\text{\text{Hom}}_{\mathbb{R}-\mathsf{Vect}}}%
\global\long\def\MT{\text{\ensuremath{\mathsf{Mor}}}_{\T}}%
\global\long\def\Open{\text{\ensuremath{\mathsf{Open}}}}%
\global\long\def\PT{\mathsf{P-Top}}%
\global\long\def\T{\mathsf{Top}}%
% Homotopy Theory specific macros

\global\long\def\Ad{\mathsf{Ad}}%
\global\long\def\Cell{\mathsf{Cell}}%
\global\long\def\Cov{\mathsf{Cov}}%
\global\long\def\Sp{\mathsf{Sp}}%
\global\long\def\Spectra{\mathsf{Spectra}}%
\global\long\def\ss{\widehat{\triangle}}%
\global\long\def\Tn{\mathbb{T}^{n}}%
\global\long\def\Sk#1{\textrm{Sk}^{#1}}%
\global\long\def\smash{\wedge}%
\global\long\def\wp{\vee}%
% HoTT specific macros

\global\long\def\base{\mathsf{base}}%
\global\long\def\comp{\mathsf{comp}}%
\global\long\def\funext{\mathsf{funext}}%
\global\long\def\hfib{\text{\ensuremath{\mathsf{hfib}}}}%
\global\long\def\I{\mathbf{I}}%
\global\long\def\ind{\mathsf{ind}}%
% LOOP as \mathsf WHAT THE FUCK!!!!

\global\long\def\lp{\mathsf{loop}}%
\global\long\def\pair{\mathsf{pair}}%
\global\long\def\pr{\mathbf{\mathsf{pr}}}%
\global\long\def\rec{\mathsf{rec}}%
\global\long\def\refl{\mathsf{refl}}%
\global\long\def\transport{\mathsf{transport}}%
%%%%%%%%%%%%%%%%%%%%%%%%%%%%%%%% UNSORTED! %%%%%%%%%%%%%%%%%%%%%%%%%%%%%%%%%%%%%%%%%%%%%%%%%%%%%

\global\long\def\is{\triangle\raisebox{2mm}{\mbox{\ensuremath{\infty}}}}%
\global\long\def\Cof{\mathsf{Cof}}%
\global\long\def\sfW{\mathsf{W}}%
\global\long\def\Cyl{\mathsf{Cyl}}%
\global\long\def\Mono{\mathsf{Mono}}%
\global\long\def\t{\triangle}%
\global\long\def\tl{\triangleleft}%
\global\long\def\tr{\triangleright}%
\global\long\def\Shift{\mathrm{Shift}_{+1}}%
\global\long\def\Shiftd{\mathrm{Shift}_{-1}}%
\global\long\def\out{\mathrm{out}}%
\global\long\def\cN{\mathcal{N}}%
\global\long\def\fC{\mathcal{\mathfrak{C}}}%
\global\long\def\ev{\mathsf{ev}}%
\global\long\def\Map{\mathsf{Map}}%
\global\long\def\whp#1{\wh{#1}_{\bullet}}%
\global\long\def\bfTwo{\mathbf{2}}%
\global\long\def\bfL{\mathbf{L}}%
\global\long\def\bfR{\mathbf{R}}%
\global\long\def\sJ{\mathscr{J}}%
\global\long\def\Sing{\mathsf{Sing}}%
\global\long\def\Sph{\mathsf{Sph}}%
\global\long\def\whfin#1{\widehat{#1}_{\mathrm{fin}}}%
\global\long\def\whpfin#1{\widehat{#1}_{\mathrm{\bullet fin}}}%
\global\long\def\fin{\mathsf{fin}}%
\global\long\def\cT{\mathcal{T}}%
\global\long\def\Alg{\mathsf{Alg}}%
\global\long\def\st{\mathsf{st}}%
\global\long\def\IN{\mathsf{in}}%
\global\long\def\hr{\mathsf{hr}}%
\global\long\def\Fun{\mathsf{Fun}}%
\global\long\def\Th{\mathsf{Th}}%
\global\long\def\sT{\mathscr{T}}%
\global\long\def\Lex{\mathsf{Lex}}%
\global\long\def\FinSet{\mathsf{FinSet}}%
\global\long\def\Fib{\mathsf{Fib}}%
\global\long\def\FPS{\mathsf{FinPos}}%
\global\long\def\Mod{\mathsf{Mod}}%
\global\long\def\sfA{\mathsf{A}}%
\global\long\def\sfV{\mathsf{V}}%
\global\long\def\inner{\mathsf{inner}}%
\global\long\def\bfI{\mathbf{I}}%
\global\long\def\Kan{\mathsf{Kan}}%
\global\long\def\Berger{\mathsf{Berger}}%
\end{quotation}
%ancient paul macros...leave them...

%macros for free prop section
\global\long\def\fF{\mathfrak{F}}%
 
\global\long\def\cG{\mathcal{G}}%
 
\global\long\def\cR{\mathcal{R}}%
 
\global\long\def\cD{\mathcal{D}}%
 %\newcommand{\Ob}{\mathsf{Ob}}
\global\long\def\evmap{\mathsf{ev}}%
 
\global\long\def\coev{\mathsf{coev}}%
 
\global\long\def\bigslant#1#2{{\raisebox{.2em}{$#1$}\left/\raisebox{-.2em}{$#2$}\right.}}%


\section{From Presentation to Type Theory and Back}

Shulman's practical type theory, hereafter $\mathsf{PTT}$, is the
natural synthesis of the previous two discussions.

\subsection{Prop to type theory}

Given a presentation 
\[
\vcenter{\vbox{\xyR{2pc}\xyC{2pc}\xymatrix{\mathfrak{F}\mathcal{R}\ar@<.25pc>[r]\ar@<-.25pc>[r] & \mathfrak{F}\mathcal{G}}
}}
\]
 of a symmetric monoidal category, we specify a type theory $\mathsf{PTT}_{\langle\mathcal{G}|\mathcal{R}\rangle}$
with:
\begin{itemize}
\item contexts $\Gamma,\Delta$ etc. being lists $x_{1}:A_{1},\dots,x_{n}:A_{n}$,
which we may write as $\left(x_{1},\dots,x_{n}\right):\left(A_{1},\dots,A_{n}\right)$,
of names for variables in generating objects $A_{i}$ drawn from the
set $G_{0}$; and
\item three sorts of judgements:
\begin{itemize}
\item term judgements, e.g 
\[
\Gamma\vdash x\ \mathsf{term}
\]
 or 
\[
\Gamma\vdash f_{\left(k\right)}\left(M\right)\ \mathsf{term};
\]
 
\item typing judgements, e.g. 
\[
x:A\vdash f\left(x\right):B
\]
corresponding to a morphism $f:A\longrightarrow B$ or
\[
x:A\vdash\left(h_{\left(1\right)}\left(x\right),h_{\left(2\right)}\left(x\right)\right):B
\]
 corresponding to a morphism $h:A\longrightarrow\left(B_{1},B_{2}\right)$
or 
\[
\vdash\left(\sd{}z^{a}\right):\left(\right)
\]
 corresponding to a scalar constant, an endomorphism $z:\left(\right)\longrightarrow\left(\right)$
of the unit object; and
\item equality (of typed terms) judgements, e.g. 
\[
x:A\vdash f\left(x\right)=h\left(g\left(x\right)\right):B
\]
corresponding to a commuting triangle 
\[
\vcenter{\vbox{\xyR{2pc}\xyC{2pc}\xymatrix{A\ar[r]^{g}\ar[dr]_{f} & C\ar[d]^{h}\\
 & B
}
}}
\]
or 
\[
x:A\vdash\left(f\left(g_{\left(1\right)}\left(x\right)\right),g_{\left(2\right)}\left(x\right)\right)=\left(h_{\left(1\right)}\left(x\right),h_{\left(2\right)}\left(x\right)\right):\left(B_{1},B_{2}\right)
\]
corresponding to a commuting triangle
\[
\vcenter{\vbox{\xyR{2pc}\xyC{2pc}\xymatrix{A\ar[r]^{g}\ar[dr]_{h} & \left(C,B_{2}\right)\ar[d]^{\left(f,\id_{B_{2}}\right)}\\
 & \left(B_{1},B_{2}\right)
}
}}
\]
 
\end{itemize}
\end{itemize}

\subsubsection{Rules for the term judgement}

The term judgments of $\mathsf{PTT}_{\langle\mathcal{G}|\mathcal{R}\rangle}$
are, derived by way of rules combining Sweedler's notation and the
generating morphisms of the signature $\mathcal{G}$, for example
we've the rule
\[
\frac{\begin{gathered}\Gamma\vdash M_{1}\quad\dots\quad\Gamma\vdash M_{n}\\
f\in\mathcal{G}\left(A_{1},\dots,A_{m};B_{1},\dots,B_{n}\right)\quad m\geq1\quad n\geq2\quad1\leq k\leq n
\end{gathered}
}{\Gamma\vdash f_{\left(k\right)}\left(M_{1},\dots,M_{n}\right)\ \mathsf{term}}
\]
which permits us to form the term $f_{\left(k\right)}\left(M_{1},\dots,M_{n}\right)$
which corresponds to the $k^{th}$ Sweedler component of a morphism
\[
f:\left(A_{1},\dots,A_{m}\right)\longrightarrow\left(B_{1},\dots,B_{n}\right)
\]
 in the signature $\mathcal{G}$. When the generator $f$ has nullary
domain, we introduce a subtly different term formation rule which
add labels drawn from an alphabet $\mathfrak{A}$, syntactic sugar
which Shulman uses to great effect later on.

\[
\frac{\begin{gathered}\\
f\in\mathcal{G}\left(;B_{1},\dots,B_{n}\right)\quad\mathfrak{a}\in\mathfrak{A}\quad n\geq2\quad1\leq k\leq n
\end{gathered}
}{\Gamma\vdash f_{\left(k\right)}^{\mathfrak{a}}\ \mathsf{term}}
\]


\subsubsection{Rules for the typing judgement}

The typing judgements of $\mathsf{PTT}_{\langle G|R\rangle}$ are
derived from rules specified by the elements of $G_{1}$, the generating
formal morphisms of the signature $\mathcal{G}$. For example, if
in the signature $\mathcal{G}$ we find a formal morphism
\[
f:\left(A_{1},\dots,A_{m}\right)\longrightarrow\left(B_{1},\dots,B_{n}\right)
\]
 then, we've a rule for the typing judgement which corresponds to
applying that morphism to a list of typed terms $\left(M_{1},\dots,M_{n}\right):\left(A_{1},\dots,A_{n}\right)$.
The foreboding vaguery will be addressed shortly.

\subsubsection{Rules for the equality judgement}

Lastly, the rules for the equality judgement assert the generating
identities as axioms and build a congruence from them.

\subsection{Type Theory to Prop}

Going the other way, to any practical type theory, $\mathsf{PTT}_{\left\langle \sd GR\right\rangle }$,
we may associate its term model, $\mathbb{\mathbf{TM}\left(\mathsf{PTT}_{\left\langle \sd{\mathcal{G}}\mathcal{R}\right\rangle }\right)}$,
the category with
\begin{itemize}
\item $\Ob\left(\mathbb{\mathbf{TM}\left(\mathsf{PTT}_{\left\langle \sd{\mathcal{G}}\mathcal{R}\right\rangle }\right)}\right)$
being the contexts of $\mathbb{\mathsf{PTT}_{\left\langle \sd{\mathcal{G}}\mathcal{R}\right\rangle }}$;
and
\item $\Mor\left(\mathbb{\mathbf{TM}\left(\mathsf{PTT}_{\left\langle \sd{\mathcal{G}}\mathcal{R}\right\rangle }\right)}\right)$
being the derivable typing judgments modulo derivable equality judgements.
\end{itemize}

\subsection{On the meaning of generation and the admissibility of structural
rules}

Since, the derivable judgements of a type theory are, in a sense,
generated by the rules of that type theory it is easy enough then
to believe that, for any presentation 
\[
\vcenter{\vbox{\xyR{2pc}\xyC{2pc}\xymatrix{\mathfrak{F}\mathcal{R}\ar@<.25pc>[r]\ar@<-.25pc>[r] & \mathfrak{F}\mathcal{G}}
}}
\]
 of a PROP $\mathcal{C}$, the term model $\mathsf{TM}\left(\mathsf{PTT}_{\langle\mathcal{G}|\mathcal{R}\rangle}\right)$
is equivalent to $\mathcal{C}$.

Indeed, Shulman shows that:
\begin{itemize}
\item the term model of the type theory$\mathsf{PTT}_{\left\langle \sd{\mathcal{G}}\emptyset\right\rangle }$
enjoys the universal property of the free PROP on a signature (\textbf{Theorem
5.17});
\item the derivable equality judgements of $\mathsf{PTT}_{\langle\mathcal{G}|\mathcal{R}\rangle}$
comprise a congruence $\sim_{R}$ on $\mathfrak{F}\mathcal{G}$ (\textbf{Proposition
6.1}); and
\item the PROP $\mathfrak{F}G_{\sim_{R}}$ is equivalent to the PROP $\mathscr{C}$
presented 
\[
\vcenter{\vbox{\xyR{2pc}\xyC{2pc}\xymatrix{\mathfrak{F}\mathcal{R}\ar@<.25pc>[r]\ar@<-.25pc>[r] & \mathfrak{F}\mathcal{G}}
}}
\]
 (\textbf{Theorem 6.2}) 
\end{itemize}
Before one makes the jump from appreciating the naturality of the
work to thinking it obvious, we must acknowledge something we've intentionally
obscured: exactly how a signature $G$ generates the rules of the
typing judgement.

The 'obvious' way define the rules of the typing judgement such that
we could expect a result like Theorem 5.17 would be to specify:
\begin{itemize}
\item a rule something like 
\[
\frac{\begin{gathered}\Gamma\vdash\left(M_{1},\dots,M_{n}\right):\left(A_{1},\dots,A_{n}\right)\\
f\in\mathcal{G}\left(A_{1},\dots,A_{n};B_{1},\dots,B_{m}\right)
\end{gathered}
}{\Gamma\vdash\left(f_{\left(1\right)}\left(M_{1},\dots,M_{n}\right),\dots,f_{\left(m\right)}\left(M_{1},\dots,M_{n}\right)\right):\left(B_{1},\dots,B_{m}\right)}
\]
which would allow us to 'apply' a morphism 
\[
f\in\mathcal{G}\left(A_{1},\dots,A_{n};B_{1},\dots,B_{m}\right)
\]
 to a term $\left(M_{1},\dots,M_{n}\right)$ of $\left(A_{1},\dots,A_{n}\right)$;
\item a rule something like 
\[
\frac{\Gamma\vdash\left(M_{1},\dots,M_{n}\right):\left(A_{1},\dots,A_{n}\right)\quad\Delta\vdash\left(N_{1},\dots N_{m}\right):\left(B_{1},\dots,B_{m}\right)}{\Gamma,\Delta\vdash\left(M_{1},\dots,M_{n},N_{1},\dots,N_{m}\right):\left(A_{1},\dots,A_{n},B_{1},\dots,B_{m}\right)}
\]
which would allow us to tensor two morphisms together; and
\item a rule something like
\[
\frac{\Gamma\vdash\left(M_{1},\dots,M_{n}\right):\left(A_{1},\dots,A_{n}\right)\quad\sigma\in S_{n}}{\Gamma\vdash\left(M_{\sigma\left(1\right)},\dots,M_{\sigma\left(n\right)}\right):\left(A_{\sigma\left(1\right)},\dots,A_{\sigma\left(n\right)}\right)}
\]
corresponding to the exchange isomorphisms permuting the generating
objects in a list thereof $\left(A_{1},\dots,A_{n}\right)$.
\[
\prftree{\Gamma\vdash M:A}{f\in\mathcal{G}\left(A;B_{1},B_{2}\right)}{B}
\]
\end{itemize}
problem with this 'obvious' way however is that derivations for typing
judgements would be non-unique. Using Shulman's example, see that
the two different associations of three composable morphisms $f:A\longrightarrow B$,
$g:B\longrightarrow C$, and $h:C\longrightarrow D$ would beget distinct
derivations of the same typing judgement.
\[
\prftree{\prftree{x:A\vdash f\left(x\right):B}{y:B\vdash g\left(y\right):C}{x:A\vdash g\left(f\left(x\right)\right):C}}{z:C\vdash h\left(z\right):D}{x:A\vdash h\left(g\left(f\left(x\right)\right)\right)}
\]
\[
\prftree{x:A\vdash f\left(x\right):A}{\prftree{y:B\vdash g\left(y\right):C}{z:C\vdash h\left(z\right):D}{y:A\vdash h\left(g\left(y\right)\right):D}}{x:A\vdash h\left(g\left(f\left(x\right)\right)\right)}
\]
As induction over derivations is far more cumbersome than induction
over derivable judgements Shulman opts for a more sophisticated tack;
Shulman defines rules for the typing judgement such that:
\begin{itemize}
\item derivations of typing judgements are unique; and
\item the structural rules, i.e. the rules corresponding to composition,
tensorings, and exchange are admissible (see Section 5)
\end{itemize}
As such, not only are Theorem 5.17, Proposition 6.1, and Proposition
6.2 provable by way of a much tamer induction, but any user of the
type theory may invoke these more naive 'structural' rules in derivations
and then freely ignore the multiplicity of derivations such reasoning
may bring about. 

\section{The Free Dual Pair}

Recall that for a vector space $V$ and its dual vector space $V^{*}$,
we've a bijection 
\[
\Hom\left(A\otimes V,B\right)\liso\Hom\left(A,V^{*}\otimes B\right)
\]
 natural in vector spaces $A$ and $B$. This example is abstracted
into the usual defintion of a dual pair in a symmetric strict monoidal
category as follows.
\begin{defn}
A dual pair $\left(D,D^{*},\coev,\ev\right)$ in a symmetric monoidal
category $\left(\mathcal{C},\left(\_,\_\right),\left(\right)\right)$
is comprised of:
\begin{itemize}
\item a pair of objects $D$ and $D^{*}$ of $\mathcal{C}$;
\item a morphism $\coev:\mathbf{1}\longrightarrow D\otimes D^{*}$; and
\item a morphism $\ev:D^{*}\otimes D\longrightarrow\mathbf{1}$
\end{itemize}
satisfying the triangle identities:
\[
\begin{split}\vcenter{\vbox{\xymatrix{D\ar@{=}[dd]\ar[dr]^{\left(\coev,D\right)}\\
 & \left(D,D^{*},D\right)\ar[dl]^{\left(D,\ev\right)}\\
D
}
}} & \vcenter{\vbox{\xymatrix{D^{*}\ar@{=}[dd]\ar[dr]^{\left(D^{*},\coev\right)}\\
 & \left(D^{*},D,D^{*}\right)\ar[dl]^{\left(\ev,D^{*}\right)}\\
D^{*}
}
}}\end{split}
\]
\end{defn}

What's more, these data clearly suggest a presentation of the PROP
generated by a dual pair. We set 
\[
\mathcal{G}=\left(\left\{ D,D^{*}\right\} ,\left\{ \coev:\left(\right)\longrightarrow\left(D,D^{*}\right),\ev:\left(D^{\star},D\right)\longrightarrow\left(\right)\right\} \right),
\]
 we set 
\[
\mathcal{R}=\left(\left\{ D,D^{*}\right\} ,\left\{ \mathsf{triangle}:D\longrightarrow D,\mathsf{triangle}^{*}:D^{*}\longrightarrow D^{*}\right\} \right),
\]
and for the maps defining the presentation we pick the ones generated
by the assignments
\[
\xyR{0pc}\xymatrix{\mathsf{triangle}\ar@{|->}[r] & \id_{D}\\
\mathsf{triangle^{*}}\ar@{|->}[r] & \id_{D^{*}}
}
\]
and
\[
\xyR{0pc}\xymatrix{\mathsf{triangle}\ar@{|->}[r] & \left(D,\ev\right)\circ\left(\coev,D\right)\\
\mathsf{triangle^{*}}\ar@{|->}[r] & \left(\ev,D^{*}\right)\circ\left(D^{*}\coev\right)
}
\]
The rules for the term judgment are few:
\begin{itemize}
\item $\vcenter{\vbox{\prftree{\left(x:A\right)\in\Gamma}{\Gamma\vdash x\ \mathsf{term}}}}$;
\item $\vcenter{\vbox{\prftree{\mathfrak{a}\in\mathfrak{A}}{1\leq k\leq2}{\coev_{\left(k\right)}^{\mathfrak{a}}\ \mathsf{term}}}}$;
and
\item $\vcenter{\vbox{\prftree{\Gamma\vdash M\ \mathsf{term}}{\Gamma\vdash N\ \mathsf{term}}{\Gamma\vdash\ev\left(M,N\right)\ \mathsf{term}}}}.$
\end{itemize}
FILL IN ADMISSIBLE RULES FOR TYPING JUDGEMENTS

note that we may derive the typing judgments corresponding the compositions
\[
\xyC{4pc}\xymatrix{\left(D\right)\ar[r]^{\left(\coev,D\right)} & \left(D,D^{*},D\right)\ar[r]^{\left(D,\ev\right)} & \left(D\right)}
\]
and 
\[
\xyC{4pc}\xymatrix{\left(D^{*}\right)\ar[r]^{\left(D^{*},\coev\right)} & \left(D,D^{*},D\right)\ar[r]^{\left(D,\ev\right)} & \left(D\right)}
\]
Using the admissible rulessing the actual rules for the type theory
(modulo the consideration of 'activeness') the canonical derivation
for that first morphism follows.
\[
\prftree{\prftree{\coev\in\mathcal{G}\left(;D,D^{*}\right)}{\mathfrak{a}\in\mathfrak{A}}{\left(132\right):\left(D,D,D^{*}\right)\liso\left(D,D^{*},D\right)}{x:D\vdash\left(\coev_{\left(1\right)}^{\mathfrak{a}},\coev_{\left(2\right)}^{\mathfrak{a}},x\right):D}}{\ev\in\mathcal{G}\left(D^{*},D\right)}{x:D\vdash\left(\sd{\coev_{\left(1\right)}^{\mathfrak{a}}}\ev\left(\coev_{\left(2\right)}^{\mathfrak{a}},x\right)\right)}
\]

Lastly, we've axioms 
\[
\prftree{}{M:D\vdash\left(\sd{\coev_{\left(1\right)}}\ev\left(\coev_{\left(2\right)},M\right)\right)=M:D}
\]
and 
\[
\prftree{}{N:D^{*}\vdash\left(\sd{\coev_{\left(2\right)}}\ev\left(N,\coev_{\left(1\right)}\right)\right)=N:D^{*}}
\]
rules enough to generate a congruence.

We've now developed a type theory for the free dual pair which endows
the dual objects $D$ and $D^{*}$ with a universal notion of element.
Since the notion of dual pair abstracted the instance of a pair of
dual vector spaces, which in particular have actual elements, it behooves
us to ask:
\begin{quotation}
``how much like a vector is an term of type $D$''
\end{quotation}
The answer is both practical and electrifying (though perhaps the
author of this blog post is too easily electrified).

It's easy enough to believe that the evaluation map 
\[
\ev:\left(D,D^{\star}\right)\longrightarrow\left(\right)
\]
 endows the terms of type $D$, or $D^{*}$ for that matter, with
structure of scalar valued functions on the other. At first glance
however, the degree to which these terms are determined by these values
is unclear. As we'll see, there is an enlightening and practical type
theoretic perspective on this question.

Consider that, for a finite dimensional vector space $V$ over a field
$k$, a basis $\left\{ \mathbf{e}_{i}\right\} _{i=1}^{n}$ for $V$
and a dual basis $\left\{ \mathbf{e}_{i}^{*}\right\} _{i=1}^{n}$
for $V^{*}$ give us an elegant way to write $\coev$ and the the
first triangle identity. We write
\[
\xyR{0pc}\xymatrix{k\ar[r]^{\coev} & V\otimes V^{*}\\
x\ar@{|->}[r] & \sum_{i=1}^{n}\mathbf{e}_{i}\otimes\mathbf{e}_{i}^{*}
}
\]
and see that
\[
\begin{split}\vcenter{\vbox{\xymatrix{V\ar@{=}[dd]\ar[dr]^{\coev\otimes V}\\
 & V\otimes V^{*}\otimes V\ar[dl]^{V\otimes\ev}\\
V
}
}} & \vcenter{\vbox{\xymatrix{\mathbf{v}\ar@{|->}[dd]\ar@{|->}[dr]\\
 & \left(\sum_{i=1}^{n}\mathbf{e}_{i}\otimes\mathbf{e}_{i}^{*}\right)\otimes\mathbf{v}\ar@{|->}[dl]\\
\mathbf{v}=\sum_{i=1}^{n}\mathbf{e}_{i}^{*}\left(\mathbf{v}\right)\cdot\mathbf{e}_{i}
}
}}\end{split}
\]
So, the first triangle identity posits that a vector is precisely
determined by its values. This property of a vector seems trivial.
Thinking more type theoretically however we see something more; the
triangle identities impose that the way in which $\ev$ endows $D$
and $D^{*}$ with the structure of collections of scalar valued functions
on the other enjoys an extensionality of sorts.

To see this more clearly, let's make a pair of notational changes
to bring the parrallel to the fore. In place of writing 
\[
\left(x,y\right):\left(D^{*},D\right)\vdash\ev\left(x,y\right):\left(\right)
\]
we'll denote $\ev$ infix by $\_\triangleleft\_$ and write
\[
\left(x,y\right):\left(D^{*},D\right)\vdash x\triangleleft y:\left(\right).
\]
Similarly, in place of writing 
\[
\vdash\left(\coev_{\left(1\right)},\coev_{\left(2\right)}\right):\left(D,D^{*}\right)
\]
 we'll denote $\coev$ by the pair $\left(u,\lambda^{D}u\right)$
and write
\[
\vdash\left(u,\lambda^{D}u\right):\left(D,D^{*}\right).
\]
With this choice of notation then, the axiom which corresponds to
the first triangle identity is 
\[
x:D\vdash\left(\sd u\lambda^{D}u\triangleleft x\right)=x:D.
\]
 Then, as Shulman points out, since $=$ is a congruence with respect
to substitution, if we've, for some term $M$, the term $\lambda^{D}u\triangleleft M$
appearing in the scalars of a list of terms, then we may replace all
instances of $u$ in the rest of the term with $M$. While a mouthful,
this is a sort of '$\beta$-reduction for duality' a relationship
between function abstraction and function evaluation. Conceptually
interesting in its own right, this obeservatin also yields a one line
proof for a familiar theorem.
\begin{lem}
(cite original result)(cyclicity of trace)Let $\left(\sC,\left(\_,\_\right),\left(\right)\right)$
be a symmetric strict monoidal category, let 
\[
\left(A,A^{*},\left(u,\lambda^{A}u\right),\_\triangleleft\_\right)
\]
 and 
\[
\left(B,B^{*},\left(v,\lambda^{B}v\right),\_\triangleleft\_\right)
\]
 be dual pairs in $\sC$, and let $f:A\longrightarrow B$ and $g:B\longrightarrow A$
be morphisms in $\sC$. Let $\mathsf{tr}\left(f\circ g\right)$ be
the composition 
\[
\xyR{0pc}\xymatrix{\left(\right)\ar[r]^{\left(v,\lambda^{B}v\right)} & \left(B,B^{*}\right)\ar[r]^{f\circ g} & \left(B,B^{*}\right)\ar[r]^{\left(12\right)} & \left(B^{*},B\right)\ar[r]^{\_\triangleleft\_} & \left(\right)}
\]
and likewise let $\mathsf{tr}\left(g\circ f\right)$ be the composition.
\[
\xyR{0pc}\xymatrix{\left(\right)\ar[r]^{\left(u,\lambda^{A}u\right)} & \left(A,A^{*}\right)\ar[r]^{g\circ f} & \left(A,A^{*}\right)\ar[r]^{\left(12\right)} & \left(A^{*},A\right)\ar[r]^{\_\triangleleft\_} & \left(\right)}
.
\]
Then, $\mathsf{tr}\left(f\circ g\right)=\mathsf{tr}\left(g\circ f\right)$.
\end{lem}

\begin{proof}
\begin{eqnarray*}
\mathsf{tr}\left(f\circ g\right) & \overset{\mathsf{def}}{=} & \left(\left|\lambda_{u}^{B}\triangleleft f\left(g\left(u\right)\right)\right.\right)\\
 & = & \left(\left|\lambda_{u}^{B}\triangleleft f\left(v\right)\right.,\lambda_{v}^{A}\triangleleft g\left(u\right)\right)\\
 & = & \left(\left|\lambda_{v}^{A}\triangleleft g\left(f\left(v\right)\right)\right.\right)\\
 & \overset{\mathsf{def}}{=} & \mathsf{tr}\left(g\circ f\right)
\end{eqnarray*}
Where the judged equalities are application of '$\beta$-reduction
for a duality'.
\end{proof}

\end{document}
